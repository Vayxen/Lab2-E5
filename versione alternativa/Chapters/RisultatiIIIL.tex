{\fontsize{12}{14}\selectfont 

Per calcolare la lunghezza del circuito dell'\emph{accessory unit} immerso nel campo magnetico, questo è stato misurato con il calibro il lato di una spira, ottenendo $(1.16 \pm 0.01) cm$. Questo valore è stato poi moltiplicato per il numero di spire (n=11), ottenendo $(12.76 \pm 0.11) cm$. %review
\\
%Poi con l'utilizzo della strumentazione sono stati presi due set di misure, uno al variare della corrente con angolo fissato a 90° ed uno al variare dell'angolo con corrente fissata a 1.5A. Oltre a questi due set è stato considerato il set della parte III come secondo set con corrente fissata ed angolo variabile. 
\\
%È stato calcolato il prodotto $LB$ facendo una media pesata e calcolandone la sigma.

%e\begin{align*}
    %LB_1 &= (5.2 \pm 0.3)\cdot 10^{-3} T \cdot m \\
    %LB_2 &= (5.31 \pm 0.17)\cdot 10^{-3} T \cdot m \\
    %LB_3 &= (5.4 \pm 0.2)\cdot 10^{-3} T \cdot m 
%\end{align*}

%Si è proceduto sostituendo al primo set il valore di $L$ ottenuto dal calibro ottenendo:

%\begin{equation*}
 %   B = (4.1 \pm 0.2) \cdot 10^{-2} T
%\end{equation*}

%Nei due set ad angolo variabile è stato sostituito questo valore $B$, ottenendo due valori di $L$; infine facendo una media pesata si è riusciti ad ottenere il risultato finale di $L$ come:

%\begin{equation*}
    %L = (12.90 \pm 0.08) cm
%\end{equation*}

\par}