{\fontsize{11}{14}\selectfont 
Gli strumenti utilizzati in questa esperienza sono:
\begin{enumerate}
    \item \textbf{Generatore DC Cosmo 3000}
    \item \textbf{Current Balance Main Unit}
    \item \textbf{Accessory Unit} con un errore di mezza tacca sulla riga graduata, $\delta_{\theta} = 0.5$°
    \item \textbf{Magnete A} largo abbastanza da poter inserire al suo interno i circuiti stampati 
    \item \textbf{Magnete B} utilizzato insieme all'accessory unit.
    \item \textbf{Sei differenti circuiti stampati} con valori tabulati. Alcuni di questi sono \emph{single length} ed i restanti sono \emph{double length}; da specificazione del manuale, i primi potrebbero risultare più corti fino a $0.2$cm ed i secondi di $0.4$cm, motivo per il quale le misure utilizzate sono state prese togliendo al valore nominale rispettivamente $0.1$cm e $0.2$cm e assegnando questi valori come errori assoluti.
    \begin{enumerate}
        \item {SF40 (1.1 $\pm$ 0.1)cm}.
        \item {SF37 (2.1 $\pm$ 0.1)cm}.
        \item {SF39 (3.1 $\pm$ 0.1)cm}.
        \item {SF38 (4.1 $\pm$ 0.1)cm}.
        \item {SF41 (6.2 $\pm$ 0.2)cm}.
        \item {SF42 (8.2 $\pm$ 0.2)cm}.
    \end{enumerate}
    \item \textbf{Multimetro analogico} con un fs in DC di 5A e con un errore del 2\% sul fs, $\delta_{I} = 0.1A$
    \item \textbf{Bilancia OHAUS Model 311} con un errore di mezza tacca sulla misura dello 0 e di un'altra mezza tacca sulla massa pesata, per un errore totale di 1 tacca: $\delta_{M} = 0.01g$
\end{enumerate}

\par}