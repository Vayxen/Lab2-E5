{\fontsize{12}{14}\selectfont 

La parte III dell'esperienza consiste nel misurare la forza di Lorentz in funzione dell'angolo $\theta$. L'apparato è stato montato come negli esperimenti precedenti, aggiungendo l'\emph{accessory unit} e rimuovendo il circuito stampato. È stato inoltre utilizzato il magnete B.
\par
Scelto come valore della corrente $(1 \pm 0.1) A$, come in precedenza è stata misurata la massa del magnete mentre l'amperometro segnava 0A ottenendo $M_{0B}' = (70.76 \pm 0.01) g$. Il generatore tuttavia riportava una corrente di 0.02A, quindi la misura è stata replicata a circuito aperto ottenendo $M_{0B} = (70.735 \pm 0.01) g$.
\par
Le misure sono state fatte ogni 20° in tutto il range di 200° dell'accessory unit. Terminato il primo set è stato ruotato il magnete di circa 180°, prendendo nuove misure con lo stesso passo. Questo ha fatto sì che i punti dei due set si sovrapponessero in un breve tratto, così da facilitare l'unione in un unico set che contenesse 2 picchi.
\par
Plottando le differenze di peso in funzione dell'angolo è stato fatto un fit per ognuno dei due set; la funzione fittata è stata $M = A\cdot \sin(\theta + \phi)$.
\\
I due set sono stati traslati della loro fase in modo da unirli in un unico set. Tuttavia i punti non si sono sovrapposti come atteso, per cui è stato ulteriormente traslato ad occhio il secondo set di misure, fino ad ottenere una perfetta sovrapposizione dei dati. Come errore su questa procedura è stato considerato $2\delta_{M}$, ovvero la distanza massima entro la quale si potevano trovare i punti. Questo errore, sommato in quadratura all'errore iniziale di 0.02g, è diventato $\delta_M = 0.04g$.

\par}