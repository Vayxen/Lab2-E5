{\fontsize{12}{14}\selectfont 

Grazie a questo esperimento è stato verificato con successo che la forza di Lorentz generata da un circuito di lunghezza $L$ ed attraversato da una corrente $I$ immerso in un campo magnetico $B$ con un angolo tra campo magnetico e circuito $\theta$ dipende linearmente dalla lunghezza del filo $L$, dalla corrente che attraversa il circuito $I$ e dal seno dell'angolo $\theta$ tra il circuito ed il campo magnetico.
\par
Ripetendo l'esperimento sarebbe opportuno prendere le varie misure di ogni set di dati in ordine aleatorio, in modo da evitare andamenti periodici negli errori dovuti a variazioni nella risposta degli strumenti col passare del tempo.
\par
Inoltre nella parte III sarebbe stato opportuno cercare di far sovrapporre più dati tra i due set, ruotando di qualche grado in meno il magnete prima di passare al set successivo, così da avere una ricucitura più efficace.


\par}
