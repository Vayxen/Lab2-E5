\documentclass[11pt, a4paper]{article}

\usepackage[margin=1in]{geometry} % manage page dimensions
\usepackage[utf8]{inputenc} % utf-8 encoding
\usepackage[italian]{babel} % italian default text
\usepackage[hidelinks]{hyperref} % link references
\usepackage{bookmark} %libro marco 
\usepackage{import} % import other .tex files
\usepackage{amsmath} % math commands
\usepackage{amssymb} % math symbols
\usepackage{amsthm} % math environments
\usepackage{amsmath} % equation align
\usepackage{siunitx} % SI unit
\usepackage{booktabs} % tabular enhance
\usepackage{multirow} % dynamic tabular cell dimensions
\usepackage{longtable} % multi-page table
\usepackage[labelfont=bf, skip=.5em, font=small]{caption} % beautiful caption
\usepackage{subcaption} % subfigure
\usepackage{graphicx} % import graphics
\usepackage{fancyhdr} % custom page header and footer

\graphicspath{{../assets/}{../graphs/}} % base graphics path

\setlength{\parskip}{1em} % distance between paragraphs
\setlength{\parindent}{0em} % indentation at beginning of paragraph

\numberwithin{equation}{section} % equation tag relative to section

\title{Esperienza di elettromagnetismo - Esperimento di Faraday}
\author{Antonio Riolo, Matteo Romano, Vittorio Strano, Florinda Tesi, Arianna Genuardi}
\date{09/03/2022} %o metto la data della prima parte dell'esperienza?

\begin{document}

\maketitle

\tableofcontents

\rule{\textwidth}{1px}

\newpage

    \section{Obiettivo dell'esperienza}

    L'obiettivo dell'esperienza è studiare l'intensità della forza di Lorentz $F_{L} = IlB \sin \theta$, ponendola ogni volta in funzione di una delle variabili e fissando le altre (eccetto $B$, che mantiene lo stesso valore per tutti e 3 i casi in esame).
    %todo: specificare qui il valore di B una volta calcolato dall'analisi dei dati, così possiamo accorciare questa parentesi

    \section{Strumentazione e accorgimenti adottati} %? review

    \begin{itemize}
        \item Circuiti stampati
        \item Bilancia ($\delta =$) %! da discutere
        \item Magneti 
        \item Generatore di corrente ($\delta = 2\% \; \text{f.s.}$) %va bene come notazione?
        \item Amperometro analogico
        \item Stand %inserisci la simpatica reference
        %\item %qui tecnicamente andrebbe la "main unit" dove si attacca il circuito stampato ma non so se chiamarla unità principale oppure ha un nome più sensato
    \end{itemize}

    %todo: parlare della taratura eseguita

    \subsection{Costruzione del circuito}

    % \begin{figure} %forse andrebbe centrato(?)
    %     \includegraphics{filename}
    %     \label{fig:circuito}
    % \end{figure}

    \section{Analisi dei dati}
    
    \subsection{Parte 1 - $F_{L}$ in funzione di $I$}

    Per la prima parte dell'esperimento, è stata misurata la variazione di massa dei magneti al variare dell'intensità di corrente. Poichè la corrente massima sopportata dall'unità principale è di $5\text{A}$, onde evitare possibili danni al circuito, si è scelto di variare $I$ tra $0\text{A}$ e $4.5\text{A}$, procedendo ad incrementi di $0.5\text{A}$. Il circuito qui utilizzato ha lunghezza $l = (8.2 \pm 0.2)cm$.

    %grafico

    %commento sul grafico

    \subsection{Parte 2 - $F_{L}$ in funzione di $l$}

    Fissato $I = (?)A$, sono stati utilizzati sei circuiti stampati di lunghezza variabile %la lunghezza diciamo che è specificata nelle ascisse dell'eventuale grafico?

    %sezione grafico

    %commento sul grafico
    %*quanto alla massa dei magneti, proporrei di non metterla subito all'inizio della sezione o specificato negli strumenti, ma subito prima di mostrare ciascun set di misure, del tipo "in questo set di misure, la massa dei magneti misurata a corrente nulla è x"

    \subsection{Parte 3 - $F_{L}$ in funzione di $\theta$}

    In quest'ultima parte di esperimento, è stato fatto variare l'angolo $\theta$ per mezzo di %un cosino per cui eventualmente ci sarà una spiegazione migliore

    %grafico

    %commento sul grafico

    \section{Conclusioni}

\end{document}