\documentclass[11pt, a4paper]{article}

\usepackage[margin=1in]{geometry} % manage page dimensions
\usepackage[utf8]{inputenc} % utf-8 encoding
\usepackage[italian]{babel} % italian default text
\usepackage[hidelinks]{hyperref} % link references
\usepackage{bookmark} %libro marco 
\usepackage{import} % import other .tex files
\usepackage{amsmath} % math commands
\usepackage{amssymb} % math symbols
\usepackage{amsthm} % math environments
\usepackage{amsmath} % equation align
\usepackage{siunitx} % SI unit
\usepackage{booktabs} % tabular enhance
\usepackage{multirow} % dynamic tabular cell dimensions
\usepackage{longtable} % multi-page table
\usepackage[labelfont=bf, skip=.5em, font=small]{caption} % beautiful caption
\usepackage{subcaption} % subfigure
\usepackage{graphicx} % import graphics
\usepackage{fancyhdr} % custom page header and footer

\graphicspath{{../assets/}{../graphs/}} % base graphics path

\setlength{\parskip}{1em} % distance between paragraphs
\setlength{\parindent}{0em} % indentation at beginning of paragraph

\numberwithin{equation}{section} % equation tag relative to section

\title{Esperienza 5}
\author{Antonio Riolo, Matteo Romano, Vittorio Strano, Florinda Tesi, Arianna } %poi inserisco il cognome, pardon ma non ricordo...
\date{09/03/2022} %o metto la data della prima parte dell'esperienza?

\tableofcontents

\begin{document}
    \section{Obiettivo dell'esperienza}

    L'obiettivo dell'esperienza è studiare l'intensità della forza di Lorentz $F_{L} = IlB \sin \theta$, ponendola ogni volta in funzione di una delle variabili e fissando le altre (eccetto $B$, che mantiene lo stesso valore per tutti e 3 i casi in esame).

    \section{Strumenti e materiali}



    \section{Analisi dei dati}
    
    \subsection{Parte 1 - FL in funzione di I}

    \subsection{Parte 2 - FL in funzione di l} %sia FL che I, l, theta verranno poi messi in math mode, intanto vedo se la struttura del documento è apposto

    \subsection{Parte 3 - FL in funzione di theta}

    \section{Conclusioni}

\end{document}